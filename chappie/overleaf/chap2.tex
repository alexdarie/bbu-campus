\chapter{Related work}
\pagestyle{fancy}
% \lhead{Related work}
\label{relatedwork}

For this section we saw fit to divide the lessons took from businesses and their already on the marker applications from the research area. One can search for weather applications making fitness recommendations, yet we decided to search for fitness tracking apps that can be extended, in our vision, through an additional analytics panel that takes into consideration the weather when predicting or making recommendations. The business layer behind these applications encloses one or more techniques of data mining, that return an appropriate response for the circumstances and the goal, openly adapting to changes in both the environment and also the goal, learning from experience and making the appropriate choice given the perceptual limitations and finite computation \cite{10.5555/1809744}. In other words, they are modeling human activities so that we can benefit from their results.

\section{Market}

New fitness trackers and smartwatches are released to the consumer market every year. These devices are equipped with different sensors, algorithms, and accompanying mobile apps. With recent advances in mobile sensor technology, privately collected physical activity data can be used as an addition to existing methods for health data collection in research. Furthermore, data collected from these devices have possible applications in patient diagnostics and treatment. \cite{info:doi/10.2196/jmir.9157}

"Wearable tech has become ingrained in today's culture, and the industry shows no signs of slowing down", said ACSM\footnote{ACSM's Health \& Fitness Journal is an official publication of the American College of Sports Medicine. Visit www.acsm-healthfitness.org for more information.} forner president Walter R. Thompson. "Tech advances have made it easier than ever for users to collect important health metrics and work with fitness professionals and health care providers to improve exercise efficiency, develop healthy lifestyles, manage chronic diseases and, ultimately, increase quality of life." Thus, it's not shocking that more than 3,000 health and fitness pros surveyed by the same publication say wearable tech will be the top trend in fitness in the coming year. \cite{WearableTrend2020}

After reading a couple of articles it is easy to observe that the wearable landscape is constantly changing as new devices are released and as new vendors enter or leave the market, or are acquired by larger vendor \cite{ForbesFitbit}. Even so, we found some trends that are reminded by multiple articles:

\begin{itemize}
    \item {Wearable devices - "there’s no doubt Google is interested in the wearables sector: it bought intellectual property related to smartwatches back in January 2019 from Fossil, though for a rather smaller figure: \$40 million. Then there’s the fact that some people think Google’s wearable operating system, Wear OS, could do with some improvement, and Fitbit has more expertise in this area than anyone except Apple. Certainly, Google has been unable to mimic in smartwatches what it did in smartphones, that is, start well behind Apple and leapfrog the Cupertino giant in terms of user base, for instance." \cite{ForbesFitbit}}
    \item {Group training and sessions with free weights - cycling and running are two of the most popular activities people integrate in their daily routine, not only because they fit in the urban environment, but also because they're connecting people, creating groups in which the members are sharing activities and engage in friendly competitions. It was "was originally conceived of as a platform to enhance the experience of cycling alone through promoting camaraderie between cyclists in different physical locations" \cite{RIVERS2020100345}}
\end{itemize}

% Other articles: 
% \cite{doi:10.1177/1541931213601247} \cite{doi:10.1080/17458927.2020.1722421} \cite{doi:10.1080/0144929X.2020.1748715}

\subsection*{Google Fit}

Google Fit is a health-tracking platform developed by Google for the Android operating system, Wear OS and Apple IOS \cite{wikigf}. It became public available on October 2014. Reaching out for information with regard to Google's product will lead us to their headline - "Coaching you to a healthier and more active life". A statement reflecting their purpose, namely a coaching, journal and tracker, all in the same place. In \textbf{collaboration with the American Heart Association}, they developed a system of "hearth points" based on their activity recommendations. These recommendations were made so that one can achieve the desired level time and energy spent on fitness activities. 

A lesson learned from this products is that, besides the gamification, it allows the users to bring data from other applications into its environment, \textbf{integrating support} for third-party apps such as "Glow Fertility - Ovulation Tracker, Period Tracker", "Calorie Counter - MyFitnessPal", "Calm - Meditate, Sleep, Relax", and so on. This help us understand how to take advantage of existing apps to better understand the customer's needs. Google Fit provides an APIs for apps and device manufacturers to manage activity data from fitness apps and sensors on Android and other devices (wearables, health monitors) \cite{wikigf}.

\subsection*{Fitbit}

Formerly Healthy Metrics Research, Fitbit was founded 13 years ago, whilst it \textbf{fueled} over \textbf{200 research studies}, or 1421 to be precise - according to Science Direct, such as investigating the impact of weight loss on breast cancer recurrence in \cite{obsfit}, or assessing physical activity and sedentary behavior in preschoolers \cite{BYUN201835}. The manufacturing line produces activity trackers, smartwatches, and wearable devices alongside a mobile application for data visualization. One of the important thing to remember about this company and its products is the experience they achieved in \textbf{perfecting the operating system} that comes with the products.

\subsection*{Samsung Health}

Samsung Health was developed to track various \textbf{aspects of daily life}, contributing to well being such as physical activity, diet, and sleep \cite{sahl}, according to its Wikipedia page. This application is not focusing on athletes, but on average people with a tendency of engaging in different types of physical activity. For example, wearing a Samsung Active 2 watch will give instant feedback when running or walking, keeping track of your time spent on the desk, without stretching, recommending to have a walk from time to time. We can say that it combines the hardware capabilities of Fitbit and the ease that made us enjoy using Google Fit.

\subsection*{Strava}

Strava is a free digital service that can be easily associated with a social-network for athletes, that has grown massively popular since its launch in 2009. It is accessible through a mobile application, web browser and most importantly for developers, through a public and pretty well documented API. Strava has two key features we are interested into:
\begin{itemize}
    \item {provides a generous amounts of information on the athlete's activity even without a smartwatch or fitness tracker connected}
    \item {takes many of the interaction-fostering features found in platforms like Instagram and Twitter, being a great base for developing new features}
\end{itemize}

\section{Research}

One search made for the key-word “strava” produced 201 relevant results on Science Direct, 84 on Scopus, 52 on Web of Science and 4 on IEEE Xplore. Aside from that, 313307 results were produced by searching  "data mining" articles. Comparatively, key-words such as “fitness”, “running” or “cycling”, “weather influence on sports” triggered almost no paper, if any, describing data mining algorithms that can be applied on data collected from mundane physical activities. Furthermore, we found no relevant results (papers, documents or books) on analysing the influence of weather on fitness habits in the context of computational systems. Those being said, this proposal has a solid background of research and engineering tools, yet almost no research paper, thus we find it worth investigating. “The number of interactive fitness technologies, applications and networks which have gamified and biomedicalized real-world activities such as cycling have increased significantly over recent years” \cite{NOBAKHT2020102509}.